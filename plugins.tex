\section{Plugins}
\label{sec:plugins}

%Coisas importantes para discutirmos:
%\begin{itemize}
%  \item \C{GST_STATIC_PAD_TEMPLATE} - OK
%  \item \C{gst_my_filter_class_init} - OK
%  \item \C{gst_myfilter_init} - OK
%  \item \C{myfilter_chain} - OK
%  \item \C{myfilter_plugin_init} - OK
%  \item \C{GST_PLUGIN_DEFINE} - OK
%  \item Definição de propriedades?
%  \item Eventos?
%\end{itemize}

Até agora só vimos como usar elementos já definidos pelo Gstreamer (e.g.
``filesrc'', ``alsasink'').  Nesta seção iremos discutir como é possível criar
novos tipos de elementos.  No Gstreamer, novos tipos de elementos podem ser
criados estendendo a classe base \emph{GstElement}.  Também é possível encapsular
esses novos elementos em plugins que podem ser distribuidos separadamente,
instalados e carregados dinamicamente. 

%Esta seção apresenta um exemplo e discute a API de plugins do Gstreamer.
%Espera-se que ao final dela, o leitor tenha uma visão geral de como criar novos
%tipos de elementos que podem ser usados em um \emph{pipeline}.

%Os plugins do Gstreamer podem ser divididos em quatro categorias principais:
%\emph{source}, \emph{sink}, \emph{transform}, and \emph{processing} plugin.

\subsection*{Elemento ``Dummy''}
Para entendermos como desenvolver um novo elemento para  o GStreamer, iremos
começar com um exemplo simples, que ao receber algum dado no seu \emph{sink
pad} irá apenas imprimir o tamanho do buffer recebido e retransmitir esses
dados em seu \emph{source pad}.  Iremos chamar esse novo elemento de
``myfilter''.  A prática de desenvolvê-lo irá nos permitir identificar as
principais funções que devem ser implementadas por um novo elemento do
Gstreamer e como encapsulá-lo em um plugin.  O código fonte desse exemplo está
divido em dois arquivo (\C{myfilter.h} e \C{myfilter.c}) e será detalhado a
seguir.

O arquivo \C{myfilter.h} (Listagem~\ref{lst:myfilter_h}) declara uma classe que
representa o novo tipo de elemento que queremos criar (\C{GstMyFilter}).  A
definição dessa classe segue o padrão do \emph{framework} GObject, e deve
estender a classe \C{GStElement} (definida no arquivo \C{gst/gst.h}).  Por
isso, é necessário criar as classes \C{GstMyFilter}~(linhas 6--9) e
\C{GstMyFilterClass}~(linhas 11--13).  A classe \C{GstMyFilter} deve estender a
classe \C{GstElement}, por isso ela define o campo \C{element} do tipo
\C{GstElement}, linha 7, e a classe \C{GstMyFilterClass} define o campo
\C{parent_class} do tipo \C{GstElementClass}, linha 12.  As linhas 15--24
definem macros padrões que usam o \emph{framework} GObject para checagem e
\C{casting} do novo tipo definido.

\lstinputlisting[
style=display,
caption={Arquivo myfilter.h},
label={lst:myfilter_h},
]{src/myfilter.h}

O arquivo \C{myfilter.c}~(Listagem~\ref{lst:myfilter_c}) implementa as
principais funções que permitem ao Gstreamer carregar instâncias deste novo
elemento e utilizá-las em \emph{pipeline}s, bem como a lógica do processamento
realizado por esse elemento.  Entre elas estão: as funções de inicialização
(\C{*_init}) e a função de processamento (\C{*_chain}).

A função \C{gst_my_filter_class_init}~(linhas 30--48) é chamada na primeira vez
que um objeto do tipo \C{GstMyFilter} for instanciado.  Essa função define
metadados para a classe, por meio do método \C{gst_class_set_static_metadata},
e define os tipos de \emph{pads} do novo elemento que estamos criando, através
das chamadas \C{gst_element_class_add_pad_template}.  As chamadas dessa função
utilizam os dados de duas estruturas do tipo \C{GstStaticPadTemplate}
(respectivamente, \C{sink_factory} e \C{src_factory}, linhas 6--16).  A macro
\C{GST_STATIC_PAD_TEMPLATE} ajuda a preencher os valores dos campos de
estruturas \C{GstStaticPadTemplate}.  Ela recebe como parâmetros: (1)~um nome
para o \emph{pad}; a direção do \emph{pad} (\C{GST_PAD_UNKNOWN},
\C{GST_PAD_SRC} ou \C{GST_PAD_SINK}); a sua informação de presença
(\C{GST_PAD_ALWAYS}, \C{GST_PAD_SOMETIMES} ou \C{GST_PAD_REQUEST}); e as
capacidades do \emph{pad}, uma estrutura do tipo \C{GstCaps *}.  Geralmente, a
inicialização de uma estrutura do tipo \C{GstCaps} é feita por meio da macro
\C{GST_STATIC_CAPS}. No nosso exemplo, linhas 10 e 16, tanto a \emph{sink pad}
como a \emph{src pad} do elemento que estamos criando recebe qualquer tipo de
dado (por isso, o parâmetro \C{"ANY"}.
%Como mencionado anteriormente, cada elemento do pipeline pode ter zero ou mais
%\emph{source pads} e zero ou mais \emph{sink pads}.  O número de \emph{pads} e
%quais são esses \emph{pads} podem ser definidos estaticamente ou dinamicamente.
%No nosso exemplo, queremos um elemento que tenha apenas um \emph{source} e um
%\emph{sink pad}.  Por isso, a definição dos \emph{pads} desse elemento é feita
%de forma estática.

A função \C{gst_my_filter_init}~(linhas 50--61) é chamada quando uma nova
%instância da classe \C{GstMyFilter} é criada.  Ela basicamente cria as
instâncias da classe \C{GstPad} relacionadas a cada um dos \emph{pads} do
elemento \emph{GstMyFilter}, chamadas \C{gst_pad_new_from_static_template}
(linhas 53 e 58).  Além disso, é também função \C{gst_my_filter_init} que a
função \C{gst_pad_set_chain_function} é registrada para ser chamada sempre que
novos dados apareçam no \emph{sink pad} do elemento ``myfilter''.

A função \C{gst_my_filter_chain}~(linhas 18--27) é a função de fato responsável
por fazer o processamento de dados relacionado com o elemento que estamos
criando.  Essa função será chamada sempre que novos dados chegarem no
\emph{sink pad} do nosso elemento.  Ela recebe um ponteiro para o \emph{pad}
(\C{GstPad*}) onde os dados chegaram, um ponteiro para o pai (\C{GstObject*}) e
um ponteiro para o buffer de dados que está chegando por esse \emph{pad}
(\C{GstBuffer *}).  No nosso exemplo, essa função apenas imprime o tamanho do
buffer recebido~(acessado por meio da chamada \C{gst_buffer_get_size(buf)}) e
reenvia esse buffer para o \emph{src pad} do elemento (usando a chamada
\C{gst_pad_push}).

%A macro \C{G_DEFINE_TYPE}, linha 4, também faz parte do
%\emph{framework} GObject e expande para a declaração das funções de
%inicialização de classe (neste caso as funções \C{gst_my_filter_init} e
%\C{gst_my_filter_class_init}).  Além disso, esta macro também expande para 
%outras funções e variáveis úteis para o \emph{framework}, tais como
%\C{*_get_type()} e \C{*_parent_class}.

As linhas 70 a 78 trazem as definições relacionadas ao encapsulamento do nosso
novo elemento em um plugin.  A macro \C{GST_PLUGIN_DEFINE} é usada para
definir o ponto de entrada (a função que será chamada quando o plugin for
carregado pelo GStreamer) e metadados para um plugin.  A macro recebe a versão
para do Gstreamer que é compatível com o  plugin (dividido em \emph{major} e
\emph{minor}), o nome do plugin, uma descrição o plugin, um ponteiro para a
função de inicialização~(no nosso caso \C{my_filter_plugin_init}).  A função 

\lstinputlisting[
style=display,
caption={Arquivo myfilter.c.},
label={lst:myfilter_c},
]{src/myfilter.c}

%\subsection*{Especificando os pads}
%Como mencionado anteriormente, a comunicação entre os elementos é gerenciada
%usando \emph{pads}.  Sendo assim, ao criar um plugin também devemos especificar
%quantos e quais tipos de \emph{pad} esse plugin aceita como entrada e saída.

%A comunicação entre os plugins pode ser classificada em duas subcategorias:
%\emph{mensagens de controle} (ou eventos) e \emph{mensagens de conteúdo de
%mídia} (ou buffers).  As mensagens de controle são cruciais para controlar o
%processamento das mídias.  As mensagens de conteúdo de mídia são responsáveis
%por transferir o conteúdo que precisa ser processado.

%\subsection*{A função plugin\_init}
%\begin{itemize}
%  \item plugin\_init
%  \item chain: Essa função é onde todo o processamento dos dados é realizado.
%        No caso de plugins simples as funções \_chain () são geralmente lineares,
%        para cada buffer na enntrada um buffer será gerado na saída também.
%        Below is a very simple implementation of a chain function: 
%  \item event
%\end{itemize}

\subsection*{Testando o novo elemento}
Para compilar o nosso plugin devemos gerar uma biblioteca dinâmica (.dll no
Windows ou .so no Linux).  O comando a seguir compila o exemplo desenvolvido
acima, gerando o arquivo \emph{myfilter.so}:

\begin{lstlisting}[style=command]
@\$@ cc -shared -fPIC \
  `pkg-config --cflags --libs gstreamer-1.0 gstreamer-base-1.0`\
  myfilter.c -o myfilter.so
\end{lstlisting}

Assumindo que o arquivo gerado \C{myfilter.so} esteja no caminho atual é
possível, por exemplo, carregar o elemento myfilter que desenvolvemos em um
\emph{pipeline} simples através da ferramenta \C{gst-launch}:

\begin{lstlisting}[style=command]
@\$@ gst-launch-1.0 --gst-plugin-path=. \
     audiotestsrc !  myfilter ! alsasink
\end{lstlisting}

Aqui estamos usando o parâmtero \C{--gst-plugin-path} para informar que o
GStreamer também deve buscar plugins no diretório corrente da linha de comando.
Como o plugin que desenvolvemos recebe qualquer tipo de dado (tipo de \emph{pad
sink} \C{ANY}, é possível utilizá-lo usá-lo em qualquer pipeline.  Por exemplo,
compare a execução do \emph{pipeline} anterior com a execução seguinte
pipeline:

\begin{lstlisting}[style=command]
@\$@ gst-launch-1.0 --gst-plugin-path=. \
     videotestsrc ! myfilter ! ximagesink
\end{lstlisting}

\subsection*{Um plugin para filtro de vídeo}
Como mencionado anteriormento, os elementos do Gstreamer são derivados da
classe base \emph{GstElement}.  Porém, quando queremos desenvolver elementos
mais complexos, pode ser interessante nos basearmos em outras classes
distribuídas nos pacotes \C{gst-plugin-base}.
A Figura~\ref{fig:plugins_class_hierachy} evidencia algumas dessas classes e
como elas estão relacionadas.
A classe \C{GstBaseSrc} é um classe base para plugins tipo \emph{source} (ou
produtores).  A classe \C{GstBaseSink} serve como base para plugins do tipo
\emph{sink} (ou consumidores).  A classe \C{GstBaseTransform} serve como base
para plugins do tipo \emph{tranform} (ou filtros).  Além disso, quando estamos
tratando especificamente de áudio ou vídeo, as classes \C{GstAudioFilter} e
\C{GstVideoFilter} também podem ser interessantes.

\tikzstyle{every node}=[draw=black,thick,anchor=west,font=\scriptsize]
\tikzstyle{selected}=[draw=red,fill=red!30]
\tikzstyle{optional}=[dashed,fill=gray!50]
\begin{figure}[H]
  \centering
  \begin{tikzpicture}[%
  grow via three points={one child at (0.5,-0.7) and
  two children at (0.5,-0.7) and (0.5,-1.4)},
  edge from parent path={(\tikzparentnode.south) |- (\tikzchildnode.west)}]
  \node {GObject}
    child {
%      node {GInitiallyUnknown}
%      child {
        node {GstElement}
        child {node{GstBaseSrc}}
        child {node{GstBaseSink}}
%        child {node{GstBin}}
        child {
          node {GstBaseTransform}
          child {node {GstAudioFilter}}
          child {node {GstVideoFitter}}
        }
%      }
    };
  \end{tikzpicture}
  \label{fig:plugins_class_hierarchy}
  \caption{Hierarquia de classes evidenciando algumas classes importantes no
           desenvolvimento de novos elementos para o GStreamer.}
\end{figure}

\lstinputlisting[
style=display,
caption={myvideofilter.h},
label={lst:myvideofilter_h},
]{src/myvideofilter.h}

\lstinputlisting[
style=display,
caption={myvideofilter.c},
label={lst:myvideofilter_c},
]{src/myvideofilter.c}

Compilando o exemplo:
\begin{lstlisting}[style=command]
@\$@ cc -shared  -fPIC \
     `pkg-config --libs --cflags gstreamer-1.0 gstreamer-base-1.0 \
                                 gstreamer-video-1.0` \
     myvideofilter.c -o myvideofilter.so
\end{lstlisting}

Exemplo de \emph{pipeline} para carregar o novo filtro: 
\begin{lstlisting}[style=command]
@\$@ gst-launch-1.0 --gst-plugin-path=. filesrc location=bunny.ogg ! \
     oggdemux ! theoradec ! videoconvert ! videoscale ! \
     video/x-raw,width=720 ! myvideofilter ! ximagesink
\end{lstlisting}


