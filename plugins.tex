\section{Plugins}
\label{sec:plugins}
Até agora só vimos como usar elementos já definidos pelo Gstreamer.  Nesta
seção iremos discutir como é possível criar novos tipos de elementos---os quais
podem ser distribuidos separadamente, instalados e usados em um
\emph{pipeline}.  Um plugin no Gstreamer nada mais é do que um tipo de ator (ou
``elemento'') que pode ser instanciado em um pipeline.

%Esta seção apresenta um exemplo e discute a API de plugins do Gstreamer.
%Espera-se que ao final dela, o leitor tenha uma visão geral de como criar novos
%tipos de elementos que podem ser usados em um \emph{pipeline}.

%Os plugins do Gstreamer podem ser divididos em quatro categorias principais:
%\emph{source}, \emph{sink}, \emph{transform}, and \emph{processing} plugin.

Os plugins do Gstreamer são derivados da classe base \emph{GstElement}.  Além
disso, outras classes distribuídas nos pacotes gst-plugin-base também podem ser
usadas para criar novos plugins.  As principais são:

\begin{itemize}
%  \item{GstElement}--a classe base mais genérica, das quais as outras classes
%                     bases derivam;
  \item{GstBaseSrc}--classe base para plugins tipo \emph{source} (ou
                     produtores);
  \item{GstBaseSink}--classe base para plugins do tipo \emph{sink} (ou
                      consumidores);
  \item{GstBaseTransform}--classe base para plugins do tipo \emph{tranform}; e
  \item{GstBin}---cria um manipuladores de plugins customizáveis similar ao
                  GstPipeline.
\end{itemize}

\tikzstyle{every node}=[draw=black,thick,anchor=west]
\tikzstyle{selected}=[draw=red,fill=red!30]
\tikzstyle{optional}=[dashed,fill=gray!50]
\begin{figure}[H]
  \centering
  \begin{tikzpicture}[%
  grow via three points={one child at (0.5,-0.7) and
  two children at (0.5,-0.7) and (0.5,-1.4)},
  edge from parent path={(\tikzparentnode.south) |- (\tikzchildnode.west)}]
  \node {GObject}
    child {
%      node {GInitiallyUnknown}
%      child {
        node {GstElement}
        child {node{GstBaseSrc}}
        child {node{GstBaseSink}}
        child {node{GstBin}}
        child {
          node {GstBaseTransform}
          child {node {GstAudioFilter}}
          child {node {GstVideoFitter}}
        }
%      }
    };
  \end{tikzpicture}
  \label{fig:class_hierarchy}
\end{figure}

\subsection*{Comunicação entre os plugins}
Como mencionado anteriormente, a comunicação entre os elementos é gerenciada
usando \emph{pads}.  Sendo assim, ao criar um plugin também devemos especificar
quantos e quais tipos de \emph{pad} esse plugin aceita como entrada e saída.

%Um \emph{pad} pode ser visto como um conector que conecta dois plugins
%(similar ao cabo de força que conecta uma tomada na parede ao DVD player).

A comunicação entre os plugins pode ser classificada em duas subcategorias:
\emph{mensagens de controle} (ou eventos) e \emph{mensagens de conteúdo de
mídia} (ou buffers).  As mensagens de controle são cruciais para controlar o
processamento das mídias.  As mensagens de conteúdo de mídia são responsáveis
por transferir o conteúdo que precisa ser processado.

\subsection*{Funções}
\begin{itemize}
  \item plugin\_init
  \item chain: Essa função é onde todo o processamento dos dados é realizado.
        No caso de plugins simples as funções \_chain () são geralmente lineares,
        para cada buffer na enntrada um buffer será gerado na saída também.
        Below is a very simple implementation of a chain function: 
  \item event
\end{itemize}

\subsection*{Exemplo de plugin}
Plugin simples contendo 1 sink pad e 1 source pad.

Apenas multiplica o dado de entrada por um número X.

\subsection*{Testando o seu plugin}
Para compilar o nosso plugin devemos gerar uma biblioteca dinâmica (.dll no
Windows ou .so no Linux).  O comando a seguire compila o exemplo desenvolvido
acima:

\begin{lstlisting}[style=command]
@\$@ cc -share -fPIC \
  `pkg-config --cflags --libs gstreamer-1.0 gstreamer-base-1.0`\
  simplefilter.c -o simplefilter.so
\end{lstlisting}

O comando a seguir carrega o plugin myfilter que desenvolvemos em um pipeline
simples:

\begin{lstlisting}[style=command]
@\$@ gst-launch-1.0 --gst-plugin-path=. \
     audiotestsrc ! audioconvert ! myfilter ! alsasink
\end{lstlisting}

\noindent compare com o resultado sem o nosso filtro:
\begin{lstlisting}[style=command]
@\$@ gst-launch-1.0 --gst-plugin-path=. \
     audiotestsrc ! audioconvert ! alsasink
\end{lstlisting}


