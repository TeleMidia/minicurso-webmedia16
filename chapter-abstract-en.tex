\begin{otherlanguage}{english}
  This short course is an introduction to GStreamer, one of the main
  free/open-source frameworks for multimedia processing.  We start
  presenting GStreamer, its architecture and the dataflow programming
  model, and then adopt a hands-on approach.  Starting with an example, a
  simple video player, we introduce the main concepts of GStreamer’s basic C
  API and implement them over the initial example incrementally, so that at
  the end of the course we get a complete video player with support for
  the usual playback operations (start, stop, pause, seek, fast-forward,
  and rewind).  We also discuss sample filters---processing elements that
  manipulate audio and video samples---and present some of the filters
  natively available in GStreamer.  Moreover, we show how one can extend the
  framework by creating a plugin with a custom filter that manipulates
  video samples.  The prerequisite for the short course is a basic
  knowledge of the C programming language.  At the end of the course, we expect
  that participants acquire a general view of GStreamer, and be able to create
  simple multimedia applications and explore its more advanced features.
\end{otherlanguage}
