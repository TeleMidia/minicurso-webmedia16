\documentclass{sig-alternate-05-2015}
\usepackage[utf8]{inputenc}
\usepackage[T1]{fontenc}
\usepackage[english,brazilian]{babel}

\begin{document}

\CopyrightYear{2016} 
\setcopyright{rightsretained} 
\conferenceinfo{WebMedia~'16}{November 08-11, 2016, Teresina, PI, Brazil} 
\isbn{978-1-4503-4512-5/16/11}
\doi{http://dx.doi.org/10.1145/2976796.2988193}

\title{Programming multimedia applications in GStreamer}
\numberofauthors{1}
\author{
\alignauthor
Guilherme F.~Lima,\;
Rodrigo C.\,M.~Santos,\;
Roberto Gerson de Albuquerque Azevedo\\
\affaddr{Department of Informatics}\\
\affaddr{PUC-Rio, Rio de Janeiro, Brazil}\\
\email{\{glima,rodrigocosta,razevedo\}@inf.puc-rio.br}
}
\maketitle
\begin{abstract}
  This short course is an introduction to GStreamer, one of the main
  free/open-source frameworks for multimedia processing.  We start
  presenting GStreamer, its architecture and the dataflow programming model,
  and then adopt a hands-on approach.  Starting with an example, a simple
  video player, we introduce each concept of GStreamer’s basic C API and
  implement it over the initial example incrementally, so that at the end of
  the course we get a complete video player with support for the usual
  playback operations (start, stop, pause, seek, fast-forward, and rewind).
  We also discuss sample filters---processing elements that manipulate audio
  and video samples.  We present the various filters nativelly available in
  GStreamer and show how one can extend the framework by creating a plugin
  with a custom filter that manipulates video samples.  The only
  prerequisite for the short course is a basic knowledge of the C
  programming language.  At the end of the short course, we expect that
  participants acquire a general view of GStreamer, and be able to create
  simple multimedia applications and explore its more advanced features.
\end{abstract}

% Generated by http://dl.acm.org/ccs.cfm ---------------------------------------
 \begin{CCSXML}
<ccs2012>
<concept>
<concept_id>10002951.10003227.10003251.10003255</concept_id>
<concept_desc>Information systems~Multimedia streaming</concept_desc>
<concept_significance>500</concept_significance>
</concept>
<concept>
<concept_id>10002951.10003227.10003251.10003256</concept_id>
<concept_desc>Information systems~Multimedia content creation</concept_desc>
<concept_significance>500</concept_significance>
</concept>
<concept>
<concept_id>10002951.10003227.10003233.10003597</concept_id>
<concept_desc>Information systems~Open source software</concept_desc>
<concept_significance>300</concept_significance>
</concept>
</ccs2012>
\end{CCSXML}
\ccsdesc[500]{Information systems~Multimedia streaming}
\ccsdesc[500]{Information systems~Multimedia content creation}
\ccsdesc[300]{Information systems~Open source software}
% End generated code -----------------------------------------------------------

\printccsdesc
\keywords{Multimedia; Digital Signal Processing; Dataflow pipeline; GStreamer, C~language; Open-source; Video player.}

\vfill
\section*{BIO}

\noindent\emph{Guilherme F.~Lima} is an associate researcher at the TeleMídia
Lab.~in PUC-Rio, Rio de Janeiro, Brazil.  His current research interests
include programming languages and models for multimedia synchronization, in
particular, the intersection between synchronous languages and multimedia.
He holds a Sc.D.~(2015) and a Sc.M.~(2011) in Informatics, and a B.A.~(2009)
in Information Systems, all from PUC-Rio.

\vskip\baselineskip
\noindent\emph{Rodrigo C.\,M.~Santos} is a PhD~Candidate in Informatics at
Pontifical Catholic University of Rio de Janeiro (PUC-Rio).  He earned a
Master's Degree in Computer Science from Federal University of Maranhão
(UFMA) in 2013.  His main research interests are inter-media
synchronization, distributed multimedia systems, reactive/synchronous
programming and multimedia programming languages.  Currently he is a
Researcher at TeleMídia Lab (PUC-Rio) and Research Intern at IBM Brazil.

\vskip\baselineskip
\noindent\emph{Roberto Gerson de Albuquerque Azevedo}
received the degree of Computer Scientist from the Federal University of
Maranhão (UFMA) (in 2008), and the degree of Master (in 2010) and a
Ph.D. (in 2015) in Informatics from PUC-Rio.  Since 2008, he works as an
associate researcher in TeleMídia Lab., where he contributes to the
specifications and reference implementation for the standards of the
Brazilian Digital TV System and ITU-T Recommendations for IPTV middleware.

\balancecolumns
\end{document}
