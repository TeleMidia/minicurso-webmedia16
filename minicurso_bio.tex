\documentclass{sig-alternate-05-2015}
\usepackage[utf8]{inputenc}
\usepackage[T1]{fontenc}
\usepackage[english,brazilian]{babel}

\begin{document}

% Copyright
\CopyrightYear{2016}
\setcopyright{rightsretained}
\conferenceinfo{WebMedia '16}{November 08-11, 2016, Teresina, PI, Brazil}
\isbn{978-1-4503-4512-5/16/11}
\doi{http://dx.doi.org/10.1145/2976796.2988193}

\title{Programming Multimedia Applications in GStreamer}
%\subtitle{[Extended Abstract]
%\titlenote{A full version of this paper is available as
%\textit{Author's Guide to Preparing ACM SIG Proceedings Using
%\LaTeX$2_\epsilon$\ and BibTeX} at
%\texttt{www.acm.org/eaddress.htm}}}
%
% You need the command \numberofauthors to handle the 'placement
% and alignment' of the authors beneath the title.
%
% For aesthetic reasons, we recommend 'three authors at a time'
% i.e. three 'name/affiliation blocks' be placed beneath the title.
%
% NOTE: You are NOT restricted in how many 'rows' of
% "name/affiliations" may appear. We just ask that you restrict
% the number of 'columns' to three.
%
% Because of the available 'opening page real-estate'
% we ask you to refrain from putting more than six authors
% (two rows with three columns) beneath the article title.
% More than six makes the first-page appear very cluttered indeed.
%
% Use the \alignauthor commands to handle the names
% and affiliations for an 'aesthetic maximum' of six authors.
% Add names, affiliations, addresses for
% the seventh etc. author(s) as the argument for the
% \additionalauthors command.
% These 'additional authors' will be output/set for you
% without further effort on your part as the last section in
% the body of your article BEFORE References or any Appendices.

\numberofauthors{1} %  in this sample file, there are a *total*
% of EIGHT authors. SIX appear on the 'first-page' (for formatting
% reasons) and the remaining two appear in the \additionalauthors section.
%
\author{
% You can go ahead and credit any number of authors here,
% e.g. one 'row of three' or two rows (consisting of one row of three
% and a second row of one, two or three).
%
% The command \alignauthor (no curly braces needed) should
% precede each author name, affiliation/snail-mail address and
% e-mail address. Additionally, tag each line of
% affiliation/address with \affaddr, and tag the
% e-mail address with \email.
%
% 1st. author
\alignauthor
Guilherme F. Lima, Rodrigo Costa, Roberto Gerson de A. Azevedo\\
       \affaddr{Department of Informatics --- PUC-Rio}\\
       \affaddr{Rio de Janeiro, Brazil}\\
       \email{\{glima, rodrigocosta, razevedo\}@inf.puc-rio.br}
}
%\date{30 July 1999}
% Just remember to make sure that the TOTAL number of authors
% is the number that will appear on the first page PLUS the
% number that will appear in the \additionalauthors section.

\maketitle
\begin{abstract}
This short course is an introduction to GStreamer, one of the main
free/open-source frameworks for multimedia processing.  We start presenting
GStreamer, its architecture and the dataflow programming model, and then adopt
a hands-on approach.  Starting with an example, a simple video player, we
introduce each concept of GStreamer’s basic C API and implement it over the
initial example incrementally, so that at the end of the course we get a
complete video player, with support for the usual playback operations (start,
stop, pause, seek, fast-forward, and rewind).  We also discuss sample
filters---elements that manipulate audio and video samples.  We present the
various filters nativelly available in GStreamer and show how one can extend
the framework by creating a plugin with a custom filter that rotates video
samples.  The only prerequisite for the short course is a basic knowledge of
the C programming language.  At the end of the short course, we expect that
participants acquire a general view of GStreamer, and be able to create simple
applications and explore its more advanced features.
\end{abstract}

%
% The code below should be generated by the tool at
% http://dl.acm.org/ccs.cfm
% Please copy and paste the code instead of the example below. 
%
\begin{CCSXML}
<ccs2012>
 <concept>
  <concept_id>10010520.10010553.10010562</concept_id>
  <concept_desc>Computer systems organization~Embedded systems</concept_desc>
  <concept_significance>500</concept_significance>
 </concept>
 <concept>
  <concept_id>10010520.10010575.10010755</concept_id>
  <concept_desc>Computer systems organization~Redundancy</concept_desc>
  <concept_significance>300</concept_significance>
 </concept>
 <concept>
  <concept_id>10010520.10010553.10010554</concept_id>
  <concept_desc>Computer systems organization~Robotics</concept_desc>
  <concept_significance>100</concept_significance>
 </concept>
 <concept>
  <concept_id>10003033.10003083.10003095</concept_id>
  <concept_desc>Networks~Network reliability</concept_desc>
  <concept_significance>100</concept_significance>
 </concept>
</ccs2012>  
\end{CCSXML}

\ccsdesc[500]{Computer systems organization~Embedded systems}
\ccsdesc[300]{Computer systems organization~Redundancy}
\ccsdesc{Computer systems organization~Robotics}
\ccsdesc[100]{Networks~Network reliability}


%
% End generated code
%

%
%  Use this command to print the description
%
\printccsdesc

% We no longer use \terms command
%\terms{Theory}

\keywords{Multimedia applications; Dataflow; Gstreamer; C language;}

\section{BIO}

\noindent\emph{Guilherme F.~Lima} é pesquisador associado do Laboratório
TeleMídia da PUC-Rio.  Seus interesses de pesquisa incluem linguagens de
programação e modelos para sincronismo multimídia.  Obteve o Doutorado em
Informática pela PUC-Rio em 2015.  Também possui Mestrado em Informática (2011)
e Bacharelado em Sistemas de Informação (2009), ambos pela PUC-Rio.

\noindent\emph{Rodrigo C.\,M.~Santos} é doutorando em Informática na PUC-Rio.
Mestre em Ciência da Computação pela Universidade Federal do Maranhão (2013).
Atualmente é pesquisador do laboratório TeleMídia/PUC-Rio e colaborador do
laboratório LAWS/UFMA, atuando principalmente na área de sistemas multimídia.

\noindent\emph{Roberto Gerson de Albuquerque Azevedo} é pesquisador associado
do Laboratório TeleMídia da PUC-Rio. Possui doutorado (2015) e mestrado (2010)
em Informática pela PUC-Rio e é Bacharel em Ciência da Computação pela
Universidade Federal do Maranhão (2008). Seus interesses de pesquisa incluem:
representação e autoria de cenas multimídia interativas; e representação,
codificação, transmissão e renderização de vídeos 3D.

%
% The following two commands are all you need in the
% initial runs of your .tex file to
% produce the bibliography for the citations in your paper.
%\bibliographystyle{abbrv}
%\bibliography{sigproc}  % sigproc.bib is the name of the Bibliography in this case
% You must have a proper ".bib" file
%  and remember to run:
% latex bibtex latex latex
% to resolve all references
%
% ACM needs 'a single self-contained file'!
%
%\balancecolumns
% That's all folks!
\end{document}
