\usepackage[T1]{fontenc}
\usepackage{textcomp}
\usepackage[utf8]{inputenc}
\usepackage[english,brazilian]{babel}
\usepackage[protrusion=true,expansion]{microtype}
\usepackage{hyperref}
\hypersetup{colorlinks=true,allcolors=black}
\usepackage{amsmath}
\usepackage{mathtools}
\usepackage{float}
\usepackage{subfigure}
\usepackage{booktabs}
\usepackage{multicol}
\usepackage{multirow}
\usepackage{natbib}
\usepackage{paralist}
\usepackage{xargs}
\usepackage[prependcaption,textsize=tiny,disable]{todonotes}
\usepackage{color}
\usepackage{soulutf8}
\setlength{\bibsep}{0.0pt}
%% -------------------------------------------------------------------------
\usepackage{tikz}
\usetikzlibrary{arrows,calc,positioning,shadows,shapes,trees}
\newdimen\hdim                  % element height
\newdimen\wdim                  % element width
\newdimen\odim                  % offset distance
\hdim=2.25em
\wdim=2.36\hdim
\odim=\hdim
\tikzstyle{element}=[
  draw=black!60,
  font={\scriptsize\sffamily},
  inner sep=0pt,
  minimum height=\hdim,
  minimum width=\wdim,
  outer sep=0pt,
  rectangle,
  rounded corners=\wdim/10,
  text centered,
  thick,
  top color=white,
  bottom color=black!20,
]
\tikzstyle{arrow}=[
  color=black!60,
  draw=black!60,
  thick,
]
\tikzstyle{arrowlabel}=[
  color=black!60,
  font={\tiny\sffamily\bfseries},
]
\tikzstyle{bin}=[
  dashed,
  draw=black!60,
  font={\tiny\sffamily\bfseries},
  rounded corners=\wdim/12,
]
\tikzstyle{binlabel}=[
  anchor=west,
  color=black!60,
  inner sep=0pt,
  outer sep=0pt,
  pos=0,
  xshift=.25em,
  yshift=-.45em,
]
\tikzstyle{state}=[
  circle,
  draw,
  font={\scriptsize\sffamily},
  inner sep=0pt,
  minimum height=1.5\hdim,
  minimum width=1.5\hdim,
  text width=\hdim,
  outer sep=0pt,
  text centered,
]
\tikzstyle{keystroke}=[
  draw,
  drop shadow={
    shadow xshift=0.25ex,
    shadow yshift=-0.25ex,
    fill=black,
    opacity=0.75,
  },
  fill=white,
  font=\scriptsize\sffamily,
  inner sep=2.5pt,
  line width=0.5pt,
  minimum width={1.7em},
  rectangle,
  rounded corners=2pt,
]
\tikzset{
  node distance=\wdim+\odim,
  >=stealth,
  shorten >=.5pt,
}
\newcommand*\keystroke[1]{%
  \tikz[baseline=(key.base)]\node[keystroke](key) {#1\strut};
}
%% -------------------------------------------------------------------------
\usepackage{listings}
\lstset{%
  basicstyle=\small\ttfamily,
  columns=fullflexible,
  keepspaces=true,
  keywordstyle=\ttfamily,
  language=C,
  escapechar={@},
  mathescape=true,
  showstringspaces=false,
  texcl=true,
  upquote=true,
  extendedchars=true,
  literate=%
    {á}{{\'a}}1%
    {é}{{\'e}}1%
    {í}{{\'i}}1%
    {ó}{{\'o}}1%
    {ú}{{\'u}}1%
    {â}{{\^a}}1%
    {ê}{{\^e}}1%
    {ô}{{\^o}}1%
    {ã}{{\~a}}1%
    {õ}{{\~o}}1%
    {ç}{{\c{c}}}1%
    {€}{\euro}1%
    {§}{\S}1%
    {°}{\textdegree{}}1%
    {ß}{{\ss}}1%
    {Ä}{{\"A}}1%
    {Ö}{{\"O}}1%
    {Ü}{{\"U}}1%
    {µ}{\textmu}1%
    {¹}{{\textsuperscript{1}}}1%
    {²}{{\textsuperscript{2}}}1%
    {³}{{\textsuperscript{3}}}1%
    {¼}{\textonequarter}1%
    {½}{\textonehalf}1%
    {¢}{\textcent}1%
    {“}{``}1%
    {”}{''}1%
    {‘}{`}1%
    {’}{'}1%
    {←}{$\leftarrow$}1%
    {→}{$\rightarrow$}1%
}
\renewcommand{\lstlistingname}{Listagem}
\newdimen\xdim
\xdim=-.5\baselineskip
\lstdefinestyle{command}{
  basicstyle=\small\ttfamily,
  aboveskip=\abovedisplayskip,
  belowskip=.5\belowdisplayskip,
}
\lstdefinestyle{display}{
  aboveskip=\abovedisplayskip,
  basicstyle=\scriptsize\ttfamily,
  belowskip=0pt,
  captionpos=b,
  frame=tb,
  numbers=left,
  numberstyle={\tiny\sffamily},
}
%% -------------------------------------------------------------------------
\def\en#1{\foreignlanguage{english}{\emph{#1}}}
\let\C\lstinline
\def\<#1>{\ensuremath{\left<#1\right>}}

\setlength{\marginparwidth}{2.5cm}
\newcommandx{\rc}[2]{{\sethlcolor{red}\hl{#1}}\todo[linecolor=red,backgroundcolor=red!25,bordercolor=red]{#2}}
\newcommandx{\rg}[1]{\todo[linecolor=cyan,backgroundcolor=cyan!25,bordercolor=cyan]{#1}}
%\newcommandx{\rghl}[2]{{\sethlcolor{cyan}\hl{#1}}\todo[linecolor=cyan,backgroundcolor=cyan!25,bordercolor=cyan]{#2}}
\newcommandx{\rghl}[2]{#1}

