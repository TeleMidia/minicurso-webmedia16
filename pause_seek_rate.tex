\section{\en{Pause}, \en{seek}, \en{fast-forward} e~\en{rewind}}
\label{sec:ops}
Nesta seção vamos incrementar o nosso primeiro exemplo 
(Listagem~\ref{lst:hello}) adicionando as operações \en{pause, seek, 
  fast-forward} e \en{rewind}. Além disso, vamos aprender como
desenvolver uma aplicação usando o GStreamer que responde à eventos de tecla.
Nossa aplicação responderá aos seguintes comandos: 
\begin{compactitem}
  \item espaço -- pausa ou resume
  \item seta direita -- avança 5 segundos
  \item seta esquerda -- retrocede 5 segundos
  \item f -- faz o vídeo tocar mais rápido (\en{fast-forward})
  \item r -- faz o vídeo tocar retrocedendo (\en{rewind})
  \item n -- faz o vídeo tocar normalmente
\end{compactitem}

Até aqui, em todos os exemplos usamos a função \C{gst_bus_timed_pop_filtered} 
para recuperar mensagens postadas no \en{bus}. Como vimos, essa função bloqueia
(ou seja, é uma função síncrona) até que uma mensagem correspondente a uma 
das máscaras passadas seja postada ou quando o tempo máximo passado é 
atingido.

O GStreamer permite que mensagens do \en{bus} sejam recuperadas de forma
assíncrona por meio de uma \en{callback}, usando o \en{loop} de eventos do
\en{framework} GLib. Usando essa forma de programação, nosso programa passa
a ser orientado a eventos, como veremos a seguir. Considere a
Listagem~\ref{lst:playbin-main}.

\lstinputlisting[
language={},
escapechar={},
style=display,
caption={Usando o \en{loop} de eventos da GLib.},
label={lst:playbin-main},
]{src/playbin_control_main.c}

Na linha~4 declaramos o ponteiro \C{GMainLoop *loop} que será utilizado para 
criar nosso \en{loop} que captura eventos e chama nossa \en{callback}. Logo
após inicializarmos o GStreamer, inicializamos esse ponteiro utilizando a
função \C{g_main_loop_new (NULL, FALSE)}, na linha~18 (poderíamos ter
chamado essa função em qualquer outro lugar do código). Essa função cria uma
instância de um \C{GMainLoop} e recebe dois parâmetros. O primeiro parâmetro
é um \C{GMainContext}, sendo este uma estrutura de dados da GLib que representa
um conjunto de fontes de eventos (\en{event sources}) que serão gerenciados 
pelo \en{loop} criado. Como não passamos nenhum contexto para essa
função (\C{NULL}), nosso \en{loop} irá gerenciar todos os eventos do
contexto \en{default}\footnotetext{Informações aprofundadas sobre como a GLib
  trata eventos utilizando o \C{GMainLoop} e \C{GMainContext} podem ser 
  encontradas em~\cite{glib}.}. O segundo parâmetro é um valor booleano
indicando se o \en{loop} criado deve começar a executar imediatamente ou não.
No nosso caso, inciaremos o \en{loop} posteriormente.

Na linha~31, utilizamos a função \C{gst_element_get_bus} para obter uma
referência para o \en{bus} do elemento \C{playbin}. Porém, ao contrário dos
outros exemplos, usamos a função
\C{gst_bus_add_watch}~(linha 32) para registrarmos
a \en{callback} \C{bus_callback} que será chamada sempre que uma nova mensagem
for postada naquele \en{bus}. O último argumento dessa função é um ponteiro
que é passado à nossa \en{callback}. No nosso exemplo, sempre que a
\en{callback} \C{bus_callback} for chamada, ela receberá o valor do ponteiro 
\C{playbin} como parâmetro (uma referência para o elemento \C{playbin} criado
na linha 20). Finalmente, a função \C{gst_bus_add_watch}
retorna um identificador para a \en{callback} registrada. Armazenamos esse
identificador na variável \C{watch_id}, que será utilizado posteriormente para
remover essa \en{callback} do \en{loop} de eventos quando não necessitarmos
mais dela.

Na linha~35 chamamos a função \C{g_main_loop_run} para executar o
\en{loop}. Essa chamada bloqueia a execução e só retorna quando o \en{loop}
termina por meio da chamada à função \C{g_main_loop_quit}. Assim, a partir
deste ponto nosso programa torna-se orientado a eventos, isto é, o fluxo do
programa é suspenso, passando a ser controlado por meio de chamadas à 
\en{callback} \C{bus_callback}. Quando o \en{loop} termina, entramos no trecho
do código de liberação de memória antes de finalizarmos o programa. Na linha~39
removemos a \en{callback} registrada (\C{g_source_remove}) e, em seguida
(linha~40), liberamos a memória alocada para o \en{loop}
(\C{g_main_loop_unref}). Vamos agora analisar o código da \en{callback} 
\C{bus_callback} (Listagem~\ref{lst:playbin-callback}).

\lstinputlisting[
language={},
escapechar={},
style=display,
caption={Callback \textit{bus\_callback}.},
label={lst:playbin-callback},
]{src/playbin_control_callback.c}

O GStreamer chama a \en{callback} \C{bus_callback} passando três parâmetros:
\C{bus}, \C{msg} e \C{data}. O primeiro parâmetro é um ponteiro para o \en{bus}
que registramos essa \en{callback}, o segundo é um ponteiro para a mensagem
postada e o último parâmetro é o valor do ponteiro passado como terceiro
argumento para a função \C{gst_bus_add_watch} (ponteiro \C{playbin}, linha~32
da Listagem~\ref{lst:playbin-main}). O tipo \C{gpointer} é um \en{alias} para
o tipo \C{void *}. A atribuição da linha~3 da
Listagem~\ref{lst:playbin-callback} é segura porque o real tipo do ponteiro
\C{data} é \C{GstElement *}, conforme a conversão explícita utilizada.
A principal parte da nossa \en{callback} é a cláusula \C{switch} nas
linhas~4--59, que testa o tipo da mensagem usando a macro \C{GST_MESSAGE_TYPE}
e executa um determinado bloco de código dependendo do valor retornado. 

Nesse exemplo, só estamos interessados em tratar três tipos de eventos postados
no \en{bus}:
\C{GST_MESSAGE_ERROR}, \C{GST_MESSAGE_EOS} e \C{GST_MESSAGE_ELEMENT}.
O primeiro tipo de mensagem é postado quando ocorre algum erro internamente
no \en{pipeline}. O segundo tipo de mensagem é postado quando o conteúdo 
exibido termina (\en{End-Of-Stream}). Um elemento pode postar mensagens 
específicas no \en{bus}, o que gera o recebimento de 
uma mensagem do tipo \C{GST_MESSAGE_ELEMENT}. Como veremos adiante,
o \en{sink} de vídeo posta uma mensagem deste tipo para informar sobre eventos
de tecla e de mouse. Os demais tipos de mensagens do GStreamer podem ser
encontradas em~\cite{gstreamer}.

Voltando ao código da Listagem~\ref{lst:playbin-callback}, vamos analisar como
cada mensagem é tratada pela \en{callback} \C{bus_callback}. A primeira
cláusula \C{case} trata mensagens do tipo \C{GST_MESSAGE_ERROR} (linhas~6--19).
Mensagens desse tipo podem conter uma \en{string} que descrevem o erro que
gerou a postagem da mensagem. Assim, nossa \en{callback} usa a função 
\C{gst_message_parse_error} (linha~11) para extrair essa \en{string} e 
imprime a mensagem na tela. Como exemplo, ao fechar a janela em que um vídeo é 
renderizado, o programa acima gera o \en{log} ilustrado na 
Listagem~\ref{lst:debug}. 

\begin{lstlisting}[
  style=display,
  caption={Exemplo de log de erro da \textit{callback} da 
    Listagem~\ref{lst:playbin-callback}},
  breaklines=true,
  label={lst:debug}
  ]
  Error: Output window was closed
  Debugging info: xvimagesink.c(555): gst_xv_image_sink_handle_xevents (): /GstPlayBin:hello/GstPlaySink:playsink/GstBin:vbin/GstXvImageSink:xvimagesink0
\end{lstlisting}

O recebimento de uma mensagem de erro leva à chamada da função 
\C{g_main_loop_quit} na linha~17. Essa chamada termina o \en{loop} da GLib, 
fazendo o programa retornar da chamada \C{g_main_loop_run} 
(Listagem~\ref{lst:playbin-main}, linha~35) e, consequentemente, parar o
\en{pipeline} e terminar a execução. A segunda cláusula \C{case} trata 
mensagens do tipo \C{GST_MESSAGE_EOS} (linhas~20--24). Quando nossa
\en{callback} é notificada que o vídeo terminou (i.e., uma mensagem desse tipo é
recebida), ela apenas chama a função \C{g_main_loop_quit}, o que tem o mesmo
efeito descrito anteriormente. Ou seja, o programa acaba quando o vídeo termina
de tocar.

Finalmente, a última cláusula \C{case} trata mensagens do tipo
\C{GST_MESSAGE_ELEMENT} (linhas~25--56). Elementos \en{sink} de vídeo emitem
mensagens desse tipo para notificar a aplicação sobre eventos de mouse e tecla
recebidos na janela criada. No jargão GStreamer, essas mensagens são chamadas
de \en{navigation messages}. Como qualquer elemento pode gerar mensagens do
tipo \C{GST_MESSAGE_ELEMENT}, utilizamos a função
\C{(gst_navigation_message_get_type} para nos indicar se a mensagem recebida
é uma \en{navigation message} do tipo \C{GST_NAVIGATION_MESSAGE_EVENT}. 
\en{Navigation messages} desse tipo indicam um evento (\C{GstEvent}) gerado 
pelo \en{sink} de vídeo que não foi tratado por nenhum elemento do
\en{pipeline}. Se esse primeiro teste for verdadeiro, chamamos a função
\C{gst_navigation_message_parse_event} que extrai esse evento e retorna 
um valor booleano indicando se essa extração ocorreu com sucesso ou não.
Se ambos os testes passarem, seguimos para tratar o evento
recebido~(linhas~31--55). 

Na linha~31 usamos a função \C{gst_navigation_event_get_type} para recuperar
o tipo do evento postado. Essa função retorna um dos seguintes valores:
\C{GST_NAVIGATION_EVENT_INVALID} (se o evento passado não for do tipo
\en{navigation}), \C{GST_NAVIGATION_EVENT_KEY_PRESS} (tecla foi pressionada),
\C{GST_NAVIGATION_EVENT_KEY_RELEASE} (tecla foi solta),  
\C{GST_NAVIGATION_EVENT_MOUSE_BUTTON_PRESS} (botão do mouse foi pressionado),  
\C{GST_NAVIGATION_EVENT_MOUSE_BUTTON_RELEASE} (botão do mouse foi solto),  
\C{GST_NAVIGATION_EVENT_MOUSE_BUTTON_MOVE} (mouse foi movimentado na janela),  
\C{GST_NAVIGATION_EVENT_COMMAND} (um comando foi enviado).  
Nessa \en{callback} estamos interessados apenas em eventos que indicam que
uma tecla foi pressionada. A função \C{gst_navigation_event_parse_key_event}
(linha~34) nos retorna qual foi a tecla pressionada. Dependendo do valor
retornado, a \en{callback} toma uma determinada ação.  

Caso a tecla seja \en{space} (espaço), o vídeo é pausado ou retomado,
dependendo do seu estado atual. A macro \C{GST_STATE} retorna o estado de um
elemento. O elemento \C{playbin} no estado \C{GST_STATE_PAUSED} nos indica que
o vídeo está pausado. Para retornar a reprodução do vídeo, usamos a função
\C{gst_element_set_state} para mudar o estado desse elemento para
\C{GST_STATE_PLAYING}. De forma semelhante, se o estado do elemento \C{playbin}
for \C{GST_STATE_PLAYING}, usamos a mesma função para mudar o estado para
\C{GST_STATE_PAUSED}, pausando o vídeo (linhas~37--40).

Passamos agora para tratar o caso da tecla pressiona ser \en{Right} ou
\en{Left} (seta direita ou esquerda, respectivamente) -- linhas~42--45.
Nesse caso, vamos avançar (seta direita) ou retroceder (seta esquerda) 10s do
vídeo. Chamamos a função \C{seek} para realizar tais operações.
A Listagem~\ref{lst:seek} mostra o código dessa função.

\lstinputlisting[
language={},
escapechar={},
style=display,
caption={Função \textit{seek}.},
label={lst:seek},
]{src/playbin_seek.c}

A função \C{seek} recebe como parâmetro um elemento e um determinado \en{offset}
que será somado ao tempo atual de reprodução. \en{Offset} positivo faz o vídeo
avançar, enquanto negativo faz o vídeo retroceder. A chamada à função
\C{gst_element_query_position} na linha~4 armazena na variável \C{cur_time} o
tempo atual de reprodução do vídeo. Na linha~5 apenas somamos o valor de
\en{offset} ao tempo atual do vídeo e armazenamos na variável \C{to_time}.
Finalmente, a função \C{gst_element_seek_simple} realize a operação \en{seek}
em um elemento. A função recebe quatro argumentos: o elemento, o formato 
para executar o \en{seek}, \en{flags} determinando opções adicionais e, por fim,
a posição para a qual se deve ir. O formato (segundo argumento) é valor do
tipo \C{GstFormat} (tipicamente usa-se o valor \C{GST_FORMAT_TIME} indicando
que queremos ir a uma posição específica em relação ao tempo do vídeo). Podemos
usar o terceiro argumento para definir alguns comportamentos que o GStreamer
deve assumir ao realizar a operação \en{seek}. O valor \C{GST_SEEK_FLAG_FLUSH}
faz com que \en{buffers} processados sejam esvaziados antes de realizar a
operação. O último argumento deve ser um valor, de acordo com o formato passado
como segundo argumento, indicando qual deve ser a nova posição do elemento
em relação ao início do fluxo. A função retorna um valor booleano indicando
se a operação foi concluída com sucesso ou não.

Voltando à Listagem~\ref{lst:playbin-callback}, vamos agora analisar o caso
da tecla pressionada ser \en{f}, \en{r} ou \en{n} (linhas~46--51). No primeiro
caso, faremos o vídeo tocar mais rápido, no segundo caso faremos o vídeo tocar
para trás e, por fim, no último caso voltaremos a taxa normal de reprodução
do vídeo. Essas três operações são realizadas por meio da chamada à função
\C{change_playback_rate}. O código dessa função é mostrado na
Listagem~\ref{lst:playback-rate}.

\lstinputlisting[
language={},
escapechar={},
style=display,
caption={Função \textit{change\_playback\_rate}.},
label={lst:playback-rate},
]{src/playbin_control_rate.c}

A mudança da taxa de reprodução de um elemento é feita por meio da chamada à
função \C{gst_element_seek} (também utilizada para realizar operações 
\en{seek} mais complexas). Essa função recebe oito parâmetros: um elemento;
a nova taxa de reprodução; um valor \C{GstFormat} (semelhante ao segundo
argumento da função \C{gst_element_seek_simple}); um conjunto opcional de 
\en{flags} (semelhante ao terceiro argumento daquela função); o tipo de
ajuste (relativo ou absoluto) do novo valor de início do fluxo (posição
de \en{seek}); a nova posição do fluxo (semelhante ao último argumento da
função anterior); o tipo de ajuste (relativo ou absoluto) do novo valor de fim
do fluxo; a nova posição de fim. Como nossa intenção aqui é apenas mudar a taxa
de reprodução, passamos a nova taxa no segundo argumento para essa função. 
Valores maiores que \C{1.0} aumentam a velocidade de reprodução, valores entre
\C{0.0} e \C{1.0} fazem o vídeo tocar em câmera lenta e valores negativos fazem
o vídeo tocar em modo \en{rewind} (``para trás'').
Para manter a continuidade do vídeo, passamos a posição atual de reprodução
(recuperada com a chamada à função \C{gst_element_query_position} na linha~4)
como novo valor de início do fluxo no sexto argumento (na prática isso implica
que não haverá mudança na posição de reprodução do fluxo). 

Dessa forma, a \en{callback} \C{bus_callback} juntamente com as funções \C{seek}
e \C{change_playback_rate} implementam o exemplo dessa seção. Para compilar
este programa, precisamos adicionar as \en{flags} de compilação do pacote
\C{gstreamer-video-1.0} (devido às funções referentes ao processamento de
mensagens do tipo \en{navigation}). Assumindo que o nome do arquivo fonte seja
\C{video_control.c}, o seguinte \en{makefile} pode ser usado para
compilar o código acima:

\lstinputlisting[
style=display,
mathescape=no,
caption={\en{Makefile} que constrói o programa ``Olá mundo''.},
label={lst:makefile},
]{src/Makefile.hello_control}
