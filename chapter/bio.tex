\documentclass{sig-alternate-05-2015}
\usepackage[utf8]{inputenc}
\usepackage[T1]{fontenc}
\usepackage[english,brazilian]{babel}
%
\makeatletter
\def\bibsize#1{\def\@bibsize{#1}}
\def\thebibliography#1{%
\ifnum\addauflag=0\addauthorsection\global\addauflag=1\fi
    \subsection*{REFERENCES}
    \@bibsize
    \list{[\arabic{enumi}]}{%
        \settowidth\labelwidth{[#1]}%
        \leftmargin\labelwidth
        \advance\leftmargin\labelsep
        \advance\leftmargin\bibindent
        \parsep=0pt plus 1pt
        \itemsep=0pt plus 1pt
        \itemindent -\bibindent
        \listparindent \itemindent
        \usecounter{enumi}
    }%
    \let\newblock\@empty
    \sloppy
    \clubpenalty4000
    \@clubpenalty \clubpenalty
    \widowpenalty4000%
    \sfcode`\.=1000\relax
}
\bibsize{\normalsize}
\makeatother
%
\begin{document}
%
\CopyrightYear{2016}%
\setcopyright{rightsretained}%
\conferenceinfo{\!\!WebMedia~'16}{November 08-11, 2016, Teresina, PI, Brazil}%
\isbn{978-1-4503-4512-5/16/11}%
\doi{http://dx.doi.org/10.1145/2976796.2988193}%
%
\title{Programming Multimedia Applications in GStreamer}
\numberofauthors{1}
\author{%
\alignauthor
Guilherme F.~Lima\quad
Rodrigo C.\,M.~Santos\quad
Roberto Gerson de Albuquerque Azevedo\\
\affaddr{Department of Informatics}\\
\affaddr{PUC-Rio, Rio de Janeiro, Brazil}\\
\email{\{glima,rsantos,razevedo\}@inf.puc-rio.br}
}
\maketitle
\begin{abstract}
  This short course is an introduction to GStreamer, one of the main
  free/open-source frameworks for multimedia processing.  We start
  presenting GStreamer, its architecture and the dataflow programming model,
  and then adopt a hands-on approach.  Starting with an example, a simple
  video player, we introduce each concept of GStreamer’s basic C API and
  implement it over the initial example incrementally, so that at the end of
  the course we get a complete video player with support for the usual
  playback operations (start, stop, pause, seek, fast-forward, and rewind).
  We also discuss sample filters---processing elements that manipulate audio
  and video samples.  We present the various filters natively available in
  GStreamer and show how one can extend the framework by creating a plugin
  with a custom filter that manipulates video samples.  The only
  prerequisite for the short course is a basic knowledge of the C
  programming language.  At the end of the short course, we expect that
  participants acquire a general view of GStreamer, and be able to create
  simple multimedia applications and explore its more advanced features.
\end{abstract}

% Generated by http://dl.acm.org/ccs.cfm ---------------------------------------
 \begin{CCSXML}
<ccs2012>
<concept>
<concept_id>10002951.10003227.10003251.10003255</concept_id>
<concept_desc>Information systems~Multimedia streaming</concept_desc>
<concept_significance>500</concept_significance>
</concept>
<concept>
<concept_id>10002951.10003227.10003251.10003256</concept_id>
<concept_desc>Information systems~Multimedia content creation</concept_desc>
<concept_significance>500</concept_significance>
</concept>
<concept>
<concept_id>10002951.10003227.10003233.10003597</concept_id>
<concept_desc>Information systems~Open source software</concept_desc>
<concept_significance>300</concept_significance>
</concept>
</ccs2012>
\end{CCSXML}
\ccsdesc[500]{Information systems~Multimedia streaming}
\ccsdesc[500]{Information systems~Multimedia content creation}
\ccsdesc[300]{Information systems~Open source software}
% End generated code -----------------------------------------------------------

\printccsdesc
\keywords{
  Multimedia;
  Digital signal processing;
  Dataflow pipeline;
  GStreamer;
  C~language;
  Open-source;
  Video player.
}


\section*{THE SHORT COURSE}

\noindent
GStreamer is currently one of the main free/open-source frameworks for
multimedia processing.  It is flexible and robust, supports a large number
of audio and video formats, and is widely used in both academy and
industry~\cite{gstreamer-apps}.  In this short course, we present the
conceptual part and the practical part of GStreamer.  In the conceptual
part, we discuss the framework's programming model, namely, the dataflow
pipeline, which is the model adopted by most systems for serious multimedia
processing, e.g., Pure Data~\cite{Puckette-M-S-2007},
CLAM~\cite{Amatriain-X-2008}, DirectShow~\cite{Chatterjee-A-1997}, etc.
Under the dataflow model, a multimedia application is structured as a
directed graph (or pipeline) in which nodes are the processing elements and
edges represent connections between elements over which audio and video
samples, and control data flow.  The dataflow model is particularly
attractive for multimedia processing because it supports implementations
that are naturally parallel, modular, and scalable~\cite{Yviquel-H-2014}.

In the practical part, we present the main concepts of GStreamer's C~API and
illustrate their use by constructing a simple video player.  Though here we
are mainly interested in video playback, i.e., decoding and presentation,
this same API can be used to capture, encode, and transmit audio and video
streams.  GStreamer natively supports a great variety of components to deal
with each of the these processing phases, and can thus be used in the
construction of applications ranging from audio and video editors,
transcoders, and transmitters, to media players.

For this reason, we believe that not only programmers, but any person
interested in multimedia processing might benefit from this short course.
The only requirements are a basic knowledge of the C~programming
language~\cite{Kernighan-B-W-1988} and familiarity with some developing
environment.  The course examples assume a GNU/Linux environment, thus some
basic Unix knowledge is also desirable though not mandatory.

The short course consists of the following eight parts.
\begin{description}
\item[Introduction to GStreamer.]  We give an overall presentation of
  GStreamer---its history, architecture, software license, notable users,
  and dependencies, plus the supported formats and platforms---and also
  discuss its programming model: the pipeline dataflow.
\item[Hello world: A video player.]  We present the first version of the
  video player example which serves as a basis for the rest of the topics
  addressed in the course.  This first version is built upon the high-level
  \texttt{playbin} API, which is one of the simplest ways to implement an
  audio or video player in GStreamer.
\item[Basic concepts: Analyzing the previous example.]  We discuss what is
  behind the apparently simple code of the previous example.  We begin by
  showing the pipeline constructed and maintained internally by the
  \texttt{playbin} element, and use this pipeline as a model to introduce
  the basic concepts of GStreamer: element, pad, caps, clock, buffer, event,
  message, bus, bin, and pipeline.  After presenting each of these concepts,
  we get back to coding and re-implement the same video player example but
  this time using GStreamer's core API.  Though one can generally rely on
  \texttt{playbin} to implement simple media players, complex applications
  demand a deep knowledge of the core API.  For this reason, from this point
  on we focus on the core API syntax and operation model.  Finally, we
  discuss the tools \texttt{gst-inspect} and \texttt{gst-launch}.  The
  former can be used to query the elements available in a particular
  installation, and the latter can be used to construct a pipeline directly
  on the command-line.
\item[Filter elements.]  We present the main audio and filters available in
  the standard GStreamer installation and discuss how these can be
  integrated into the video player example.  At this point, we present the
  result of combining the example with different filters, and also discuss
  how filter parameters can be dynamically modified, at run-time.
\item[Input and output.]  We add support to keyboard and mouse input to the
  video player example, and also show how to redirect the output video
  samples to a specific OS-level window---up to this point, the example was
  rendering on a new window created automatically by one of the pipeline
  elements.
\item[Pause, seek, fast-forward and rewind.]  We discuss the theory and
  implementation (over the video player example) of the usual playback
  operations: pause, seek, fast-forward and rewind.  We also discuss the
  problems involved in supporting each of these operations and the
  situations in which they may fail.
\item[Plugins.]  We present GStreamer's plugin architecture and implement a
  custom video filter element.  Besides show\-ing its code, we discuss how
  a custom element can be wrapped into a plugin and installed in the
  system, and how it can be used by other GStreamer applications, in
  particular, how it can be used by the original video player example.
\item[Conclusion.]  We discuss briefly some advanced topics, such as dynamic
  changes in pipeline topology, the implementation of sample mixers,
  inter-pipeline synchronization, audio and video capturing, encoding and
  decoding, transmission on network, and language bindings, and conclude the
  course by pointing out references for further study.
\end{description}


\section*{BIO}

\noindent\emph{Guilherme F.~Lima} is an associate researcher at the TeleMídia
Lab.~in PUC-Rio, Rio de Janeiro, Brazil.  His research interests include
programming languages and models for multimedia synchronization, in
particular, those occurring in the intersection between synchronous
languages and multimedia.  He holds a Sc.D.~(2015) and a Sc.M.~(2011) in
Informatics, and a B.A.~(2009) in Information Systems, all from PUC-Rio.

\vskip\baselineskip
\noindent\emph{Rodrigo C.\,M.~Santos} is a PhD~Candidate in Informatics at
Pontifical Catholic University of Rio de Janeiro (PUC-Rio).  He earned a
Master's Degree in Computer Science from Federal University of Maranhão
(UFMA) in 2013.  His main research interests are inter-media
synchronization, distributed multimedia systems, reactive/synchronous
programming and multimedia programming languages.  

\vskip\baselineskip
\noindent\emph{Roberto Gerson de Albuquerque Azevedo} is an associate
researcher at TeleMídia Lab/PUC-Rio.  He earned his Ph.D. (2015) and Masters
(2010) degrees in Informatics from PUC-Rio.  He also received the degree of
Computer Scientist from the Federal University of Maranhão (UFMA)
(2008).  His research interests include authoring and representation of
multimedia scenes; and 3D video representation, coding, transmission, and
rendering.

\bibliographystyle{plain}
\bibliography{bib}
\end{document}
