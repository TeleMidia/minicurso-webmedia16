\section{Conclusão}
\label{sec:conclusao}
Neste minicurso apresentamos uma visão introdutória do \en{framework}
GStreamer.  Discutimos o modelo de \en{dataflow}, no qual ele se baseia, e como
esse modelo é implementado pelo \en{framework}.  Por meio da implementação de
exemplos mostramos como se dá o processo de desenvolvimento de aplicações no
GStreamer, e como é possível estendê-lo. 
%O GStreamer permite que dados sejam adicionados ou removidos do \en{pipeline}
%por meio dos elementos \en{appsrc} e \en{appsink}, respectivamente. \en{Appsrc}
%é um elemento do tipo \en{source} e é usado para injetar dados customizados em
%um \en{pipeline}. Um típico exemplo de uso desse elemento é na renderização
%de um formato de dados não suportado nativamente pelos elementos disponíveis
%no GStreamer (essa funcionalidade também pode ser implementada por meio da
%criação de um novo elemento, como ilustrado na Seção~\ref{sec:plugins}).
%\en{Appsink} é um elemento do tipo \en{sink} usado para receber os \en{buffers}
%de saída de uma aplicação. Um cenário de uso para o \en{appsink} é quando se
%deseja embutir a saída de um \en{pipeline} em outra uma outra aplicação.

Apenas com o que se aprendeu neste minicurso e, facilmente usando outros
elementos já suportados pelo GStreamer, já é possível que os leitores criem
várias aplicações mais avançadas.  Por exemplo, é possível compor diversos
vídeos e/ou áudios em um mesmo \en{pipeline}.  Elementos \en{mixers}
recebem vários fluxos de um mesmo tipo e geram uma única saída representando a
composição daqueles fluxos. Exemplos de \en{mixers} de vídeo são os elementos
\en{videomix} e \en{compositor}, enquanto os elementos \en{adder} e
\en{audiomixer} são exemplos de \en{mixers} de áudio.  Outro caso de uso
interessante é o uso do GStreamer para transcodificar mídias.  Podemos
facilmente criar \en{pipelines} que combinem elementos que decodifiquem um
determinado formato produzindo um fluxo \en{raw} com elementos que recebem esse
fluxo \en{raw} e o codifiquem em outro formato.  Nesses casos, o \en{sink}
\en{filesink} pode ser usado para escrever a saída do pipeline em um arquivo.

Ainda assim, vários tópicos avançados não foram apresentados aqui.  No restante
desta seção discutimos brevemente alguns desses tópicos com o objetivo de
direcionar os leitores em aprofundamentos futuros.

Um dos temas mais desafiadores em sistemas multimídia, sem dúvida é
transmissão/recepção de dados multimídia em tempo real.  Embora não tratados
aqui, o GStreamer tem vários elementos que dão suporte à implementação de
aplicações multimídia em rede. Por exemplo, o pacote \en{gst-plugins-good}
implementa vários elementos que podem ser usados para criar servidores de
\en{streaming} RTP/RTCP/RTSP. Em linhas gerais, um servidor RTP deve empacotar
fluxos codificados em um dado formato em pacotes RTP e enviá-los por meio do
protocolo UDP, como ilustrado no \en{pipeline} da Figura~\ref{fig:rtp-server}.
O elemento \en{rtptheorapay} recebe um fluxo de vídeo Theora e o encapsula em
pacotes RTP que são enviados pela rede pelo elemento \en{udpsink} usando o
protocolo UDP. De forma análoga, um cliente para esse servidor poderia ser
implementado usando o elemento \en{udpsrc}, que recebe um fluxo de dados UDP,
conectado ao elemento \en{rtptheoradepay}, que desempacota os pacotes RTP
gerando como saída um fluxo de vídeo Theora, como ilustrado no \en{pipeline} da
Figura~\ref{fig:rtp-client}. Além de elementos que encapsulam e desencapsulam
formatos específicos de fluxos em pacotes RTP, no GStreamer há diversos outros
elementos que facilitam a implementação de aplicações de \en{streaming}, como
por exemplo \en{rtpmanager}, \en{rtpbin}, \en{rtpjitterbuffer} etc.

\begin{figure}[H]
  \centering
  \subfigure [Servidor RTP]
  {
    \begin{tikzpicture}
      \node (filesrc) [element] {filesrc};
      \node (oggdemux) [element, right of=filesrc] {oggdemux};
      \node (rtptheorapay) [element, right of=oggdemux] {rtptheorapay};
      \node (udpsink) [element, right of=rtptheorapay] {udpsink};
      \draw [->, arrow] (filesrc) -- (oggdemux);
      \draw [->, arrow] (oggdemux) -- (rtptheorapay);
      \draw [->, arrow] (rtptheorapay) -- (udpsink);
    \end{tikzpicture}
    \label{fig:rtp-server}
  }
  \subfigure [Cliente RTP]
  {
    \begin{tikzpicture}
      \node (udpsrc) [element] {udpsrc};
      \node (rtptheoradepay) [element, right of=udpsrc] {rtptheoradepay};
      \node (theoradec) [element, right of=rtptheoradepay] {theoradec};
      \node (xvimagesink) [element, right of=theoradec] {xvimagesink};
      \draw [->, arrow] (udpsrc) -- (rtptheoradepay);
      \draw [->, arrow] (rtptheoradepay) -- (theoradec);
      \draw [->, arrow] (theoradec) -- (xvimagesink);
    \end{tikzpicture}
    \label{fig:rtp-client}
  }
  \caption{\en{Pipeline}s de um servidor e cliente RTP para vídeos 
    OGG codificados no formato Theora.}
  \label{fig:pipe-rtp-server}
\end{figure}

Outro tema importante, e relacionado tanto com aplicações locais como com
aplicações em rede, diz respeito à QoS.  Neste minicurso, esse tema foi
discutido apenas brevemente na Seção~\ref{sec:intro}.  Porém, o GStreamer possui várias
funcionalidades avançadas com o objetivo de manter a QoS de aplicações
multimídia.  Por exemplo, internamente, \en{pipelines} mantém um relógio do
tipo \C{GstClock} usado para sincronizar a exibição dos \en{buffers}.
Elementos \en{sink} usam esse relógio para controlar a renderização dos dados,
bem como para descartar \en{buffers} que chegam atrasados. É possível
sincronizar o relógio de \en{pipelines} diferentes (mesmo em máquinas
separadas) por meio dos objetos \C{GstNetTimeProvider} (expõe o tempo de um
\C{GstClock} para a rede) e \C{GstNetClientClock} (sincroniza o relógio de um
\en{pipeline} a um objeto provedor de tempo -- \C{GstNetTimeProvider}), por
exemplo. É possível ainda criar relógios sincronizados com um servidor NTP
(\C{GstNtpClock}) ou PTP (\C{GstPtpClock}).  Usando tais mecanismos, é possível
usar o GStreamer como base para o desenvolvimento de aplicações multimídia
avançadas, tais como aplicações que requerem a sincronização de mídias em
múltiplos destinos, bem como aplicações que requerem a sincronização de
mídias vindas de múltiplos \en{sources}~(por exemplo, cenários de TV digital
híbridas)~\cite{Cesar-2016}.

Este minicurso, obviamente, não teve a pretensão de exaurir todas essas
funcionalidades do GStreamer, mas sim de servir como um material introdutório.
Consideramos que os leitores que tenham conseguido assimilar os conceitos aqui
apresentados estejam aptos a se aventurar em outros tópicos mais avançados.  No
site oficial do framework~\cite{gstreamer} há diversos materiais detalhando o
funcionamento interno do GStreamer, bem como alguns tópicos não abordados aqui.
Apesar de não ser essencial, àqueles que querem se aventurar a trabalhar 
diretamente no código do GStreamer, ou mesmo estendê-lo por meio de
\en{plugins}, noções do \en{framework} GObject~\cite{glib} serão bem úteis.  
Mais ainda, caso C não seja a sua linguagem favorita, existem também 
vários~\en{bindings} do GStreamer para outras
linguagens\footnote{\url{https://gstreamer.freedesktop.org/bindings/}}---tais 
como Python, Java, C++, Qt, Android, Vala, Ruby, Haskell etc---por meio das
quais você pode tirar proveito do poder do GStreamer.

